\documentclass[a4paper,10pt,ngerman]{scrartcl}
\usepackage{babel}
\usepackage[T1]{fontenc}
\usepackage[utf8]{inputenc}
\usepackage[a4paper,margin=2.5cm]{geometry}

% Die nächsten drei Felder bitte anpassen:
\newcommand{\Name}{Richard Wohlbold} % Teamname oder eigenen Namen angeben
\newcommand{\TeamId}{00012}                       % Team-ID aus dem PMS
\newcommand{\Aufgabe}{Aufgabe 1: Superstar} % Aufgabennummer und -name

% Kopf- und Fußzeilen
\usepackage{scrlayer-scrpage}
\setkomafont{pageheadfoot}{\textrm}
\ifoot{\Name}
\cfoot{\thepage}
\chead{\Aufgabe}
\ofoot{Team-ID: \TeamId}

% Für mathematische Befehle und Symbole
\usepackage{amsmath}
\usepackage{amssymb}

% Für Bilder
\usepackage{graphicx}

% Für Algorithmen
\usepackage{algpseudocode}

% Für Quelltext
\usepackage{listings}
\usepackage{color}
\definecolor{mygreen}{rgb}{0,0.6,0}
\definecolor{mygray}{rgb}{0.5,0.5,0.5}
\definecolor{mymauve}{rgb}{0.58,0,0.82}
\lstset{
  keywordstyle=\color{blue},commentstyle=\color{mygreen},
  stringstyle=\color{mymauve},rulecolor=\color{black},
  basicstyle=\footnotesize\ttfamily,numberstyle=\tiny\color{mygray},
  captionpos=b, % sets the caption-position to bottom
  keepspaces=true, % keeps spaces in text
  numbers=left, numbersep=5pt, showspaces=false,showstringspaces=true,
  showtabs=false, stepnumber=2, tabsize=2, title=\lstname
}

% Diese beiden Pakete müssen als letztes geladen werden
%\usepackage{hyperref} % Anklickbare Links im Dokument
\usepackage{cleveref}

% Daten für die Titelseite
\title{\Aufgabe}
\author{\Name\\Team-ID: \TeamId}
\date{\today}



\begin{document}

\maketitle
\tableofcontents
\clearpage

\section{Lösungsidee}
Mein Verfahren, um zu ermitteln, ob es einen Superstar gibt und wie dieser heißt ist in zwei
Phasen unterteilt:
\begin{enumerate}
\item In der \textbf{Ermittlungsphase} wird aus der gesamten Gruppe genau eine Person durch ein Ausschlussverfahren
ermittelt, die ein Superstar sein könnte. 
\item In der \textbf{Validierungsphase} wird überprüft, ob die in der Ermittlungsphase 
ermittelte Person tatsächlich ein Superstar ist.
\end{enumerate}

\paragraph{Ermittlungsphase}
Zur Ermittlung, ob es einen Superstar gibt und wie dieser eventuell heißt, benutze ich
ein Ausschlussverfahren. Am Anfang sind alle Mitglieder der Gruppe mögliche Superstars,
da bisher keine Informationen vorliegen, die beweisen, dass Mitglieder keine Superstars
sind.
Um genau einen möglichen Superstar aus wenig Anfragen zu bestimmen, muss man aus jeder Anfrage
so viele Informationen wie möglich gewinnen.
Wenn $X$ und $Y$ zwei verschiedene Mitglieder einer Gruppe sind und beide mögliche Superstars
sind, so schließt eine Anfrage der Form "Folgt $X$ $Y$?" immer entweder $X$ oder $Y$ aus der
Liste der möglichen Superstars aus, denn:
\begin{itemize}
  \item Falls $X$ $Y$ folgt, so kann $X$ kein Superstar sein, denn ein Superstar folgt keinem
  anderen Mitglied der Gruppe
  \item Falls $X$ nicht $Y$ folgt, so kann $Y$ kein Superstar sein, denn alle anderen Mitglieder
  der Gruppe folgen einem Superstar
\end{itemize}
Um nun die Anzahl der möglichen Superstars auf einen zu reduzieren, nimmt man die ersten
zwei Mitglieder $X$ und $Y$ aus der Liste, stellt die Anfrage \textit{Folgt $X$ $Y$?} und fügt
das nichtausgeschlossene Mitglied wieder der Liste hinzu.\newline
Da mit diesem Verfahren mit einer Anfrage je ein Mitglied aus der Liste der möglichen Superstars
eliminiert werden kann und am Anfang $n$ Mitglieder in der Liste sind und am Ende nur $1$ Mitglied
in der Liste verbleibt, müssen $n-1$ Anfragen gestellt werden, um die Anzahl der möglichen
Superstars von $n$ auf $1$ zu reduzieren.

\paragraph{Validierungsphase}
Da jetzt nurnoch ein möglicher Superstar $S$ vorliegt, muss noch überprüft werden, ob dieser tatsächlich
einer ist. Die einzige Möglichkeit, um dies zu überprüfen, ist für jedes andere Mitglied $X$ zwei Anfragen
zu stellen:
\begin{enumerate}
\item Folgt $X$ $S$? Falls nein, ist $S$ kein Superstar.
\item Folgt $S$ $X$? Falls ja, ist $S$ kein Superstar.
\end{enumerate}
Falls bei einer der Anfragen herauskommt, dass $S$ kein Superstar ist, ist die Validierungsphase
beendet und es gibt keinen Superstar in der Gruppe, da $S$ der einzige mögliche Superstar
vor der Validierungsphase war.\newline
Falls bei keiner der Anfragen herauskommt, dass $S$ kein Superstar ist, haben wir festgestellt,
dass $S$ niemandem folgt, jedoch $S$ jeder folgt, sodass $S$ ein Superstar ist.\newline
Falls $S$ ein Superstar ist, werden in der Validierungsphase werden $n-1$ Anfragen der Form 
\textit{Folgt $S$ $X$?} und $n-1$ Anfragen der Form \textit{Folgt $X$ $S$?} gestellt, sodass maximal $2(n-1)$
Anfragen gestellt werden.

\paragraph{Optimierung des Verfahrens}
Da in der Validierungsphase alle möglichen Anfragen, die den möglichen Superstar $S$
enthalten, gestellt werden müssen, aber schon in der Validierungsphase einige dieser Anfragen
gestellt werden müssen, um $S$ zu ermitteln, kommt es zwingend vor, dass Anfragen doppelt
gestellt werden. Da jede doppelt gestellte Anfrage zu vermeiden ist, können bereits gestellte 
Anfragen und ihre Ergebnisse gespeichert werden und für jede Anfrage kann überprüft werden,
ob ihr Ergebnis bereits im Zwischenspeicher existiert, sodass die Anzahl der Anfragen leicht gesenkt
werden kann. Da in der Informatik jedoch oft der Worst-Case zählt, ist meine Schätzung der Anzahl der Anfragen $3(n-1) = 3n-3$.

\clearpage


\section{Umsetzung}
Das Verfahren zur Bestimmung eines Superstars wurde in Python umgesetzt.\newline
Dazu wird als erstes die Eingabe in die Liste der Mitglieder und die Liste der Verbindungen
($X$ folgt $Y$) eingelesen. Dabei ist eine Verbindung eine Liste der beiden Namen der Form
\texttt{[X, Y]}. Dabei kümmert sich die Funktion \texttt{folgt} um das Stellen der Anfrage,
das Speichern, falls die gleiche Anfrage noch einmal gestellt werden sollte und um das Ausgeben
der Anfrage und des Ergebnisses. Der Zwischenspeicher ist ein \texttt{dict} das als Schlüssel
Tupel der Form \texttt{(X,Y)} verwendet, da Python keine Listen als Schlüssel unterstützt,
und als Wert, ob $X$ $Y$ folgt, zurückgibt. 

\clearpage

\section{Beispiele}
Die erste Zeile ist nicht Teil des Programms, sondern dient zur Illustration, wie das Skript aufgerufen werden sollte.
Die letzte leere Zeile und das \$ soll zeigen, dass die Ausgabe hier endet.\newline

Beispiel 1 von der bwinf-Webseite:
\begin{lstlisting}
$ python superstar.py ../beispieldaten/superstar1.txt
Folgt Hailey Justin? 	Ja!
Folgt Justin Selena? 	Nein!
Folgt Selena Justin? 	Ja!
Folgt Justin Hailey? 	Nein!
Superstar: Justin
3 Mitglieder
4 gestellte Anfragen

$
\end{lstlisting}

Beispiel 2 von der bwinf-Webseite:
\begin{lstlisting}
$ python superstar.py ../beispieldaten/superstar2.txt
Folgt Codd Knuth? 	Ja!
Folgt Knuth Dijkstra? 	Ja!
Folgt Dijkstra Hoare? 	Nein!
Folgt Dijkstra Turing? 	Nein!
Folgt Turing Dijkstra? 	Ja!
Folgt Hoare Dijkstra? 	Ja!
Folgt Dijkstra Knuth? 	Nein!
Folgt Codd Dijkstra? 	Ja!
Folgt Dijkstra Codd? 	Nein!
Superstar: Dijkstra
5 Mitglieder
9 gestellte Anfragen

$
\end{lstlisting}

Beispiel 3 von der bwinf-Webseite:
\begin{lstlisting}
$ python superstar.py ../beispieldaten/superstar3.txt
Folgt Sjoukje Rinus? 	Nein!
Folgt Sjoukje Rineke? 	Nein!
Folgt Sjoukje Pia? 	Nein!
Folgt Sjoukje Peter? 	Nein!
Folgt Sjoukje Jorrit? 	Nein!
Folgt Sjoukje Jitse? 	Nein!
Folgt Sjoukje Edsger? 	Ja!
Folgt Jitse Edsger? 	Ja!
Folgt Edsger Jitse? 	Ja!
Kein Superstar
8 Mitglieder
9 gestellte Anfragen

$
\end{lstlisting}

Beispiel 4 von der bwinf-Webseite (Die Liste der Anfragen wurde hier 
aus Platzgründen auf die erste und lezte Zeile reduziert):
\begin{lstlisting}
$ python superstar.py ../beispieldaten/superstar4.txt
Folgt Rut Penny? 	Nein!
...
Folgt Folke Rut? 	Nein!
Superstar: Folke
80 Mitglieder
211 gestellte Anfragen

$
\end{lstlisting}

\clearpage

\section{Quellcode (ausschnittsweise)}

Die Funktion \texttt{superstar\_bestimmen}, die den Superstar ermittelt,
falls es einen gibt:
\lstinputlisting[language=Python]{superstar_bestimmen.py}

\vspace{30pt}

Die Funktion \texttt{folgt}, die eine Anfrage stellt, sie zwischenspeichert,
die Anfrage und das Ergebnis ausgibt und die Gesamtkosten verfolgt:
\lstinputlisting[language=Python]{folgt.py}


\end{document}
