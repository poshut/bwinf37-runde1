\documentclass[a4paper,10pt,ngerman]{scrartcl}
\usepackage{babel}
\usepackage[T1]{fontenc}
\usepackage[utf8]{inputenc}
\usepackage[a4paper,margin=2.5cm]{geometry}

\usepackage[backend=biber]{biblatex}
\addbibresource{schrebergaerten.bib}

% Die nächsten drei Felder bitte anpassen:
\newcommand{\Name}{Richard Wohlbold} % Teamname oder eigenen Namen angeben
\newcommand{\TeamId}{00012}                       % Team-ID aus dem PMS
\newcommand{\Aufgabe}{Aufgabe 4: Schrebergärten} % Aufgabennummer und -name

% Kopf- und Fußzeilen
\usepackage{scrlayer-scrpage}
\setkomafont{pageheadfoot}{\textrm}
\ifoot{\Name}
\cfoot{\thepage}
\chead{\Aufgabe}
\ofoot{Team-ID: \TeamId}

% Für mathematische Befehle und Symbole
\usepackage{amsmath}
\usepackage{amssymb}

% Für Bilder
\usepackage{graphicx}

% Für Algorithmen
\usepackage{algpseudocode}

% Für Quelltext
\usepackage{listings}
\usepackage{color}
\definecolor{mygreen}{rgb}{0,0.6,0}
\definecolor{mygray}{rgb}{0.5,0.5,0.5}
\definecolor{mymauve}{rgb}{0.58,0,0.82}
\lstset{
  keywordstyle=\color{blue},commentstyle=\color{mygreen},
  stringstyle=\color{mymauve},rulecolor=\color{black},
  basicstyle=\footnotesize\ttfamily,numberstyle=\tiny\color{mygray},
  captionpos=b, % sets the caption-position to bottom
  keepspaces=true, % keeps spaces in text
  numbers=left, numbersep=5pt, showspaces=false,showstringspaces=true,
  showtabs=false, stepnumber=2, tabsize=2, title=\lstname
}

% Diese beiden Pakete müssen als letztes geladen werden
%\usepackage{hyperref} % Anklickbare Links im Dokument
\usepackage{cleveref}

% Daten für die Titelseite
\title{\Aufgabe}
\author{\Name\\Team-ID: \TeamId}
\date{\today}

\begin{document}

\maketitle
\tableofcontents
\clearpage


\section{Lösungsidee} 

\subsection{Allgemeines zum Problem}
Das Problem, mehrere Rechtecke unterschiedlicher Größe in eines zu packen, ist schon länger bekannt, da es Anwendung im Internet findet, z.B. wenn mehrere kleine Bilddateien in eine möglichst kleine gepackt werden sollen. Es ist auch bekannt, dass das Problem $NP$-komplett ist, d.h. es gibt keine optimale Lösung, die in Polymomialzeit läuft, man kann eine optimale Lösung also nur durch Ausprobieren aller Möglichkeiten bestimmen \cite{korf}. Deshalb kann jede effiziente Lösung nur ein \textit{fast} perfektes Ergebnis produzieren; mann muss also zwischen Laufzeit und Genauigkeit abwägen.

\subsection{Das Verfahren}

Da das Problem schon länger bekannt ist, benutze ich auch zum Teil ein bekanntes Verfahren: \textit{Next-Fit-Descreasing-Height} (\textit{NFDH}): Alle Rechtecke werden absteigend nach Länge sortiert. Ein großes Rechteck mit unendlicher Länge und der größten Breite, die ein einzelner Garten hat, wird erstellt. Für jedes Rechteck $R$ wird überprüft, ob $R$ von der Breite noch in die momentane Ebene passt. Falls dies nicht der Fall ist, wird eine neue Ebene mit der Länge von $R$ erstellt. Da alle Rechtecke nach $R$ eine geringere Länge als $R$ haben, passen sie immer von der Länge in die momentane Ebene. Wenn alle Rechtecke im großen Rechteck platziert sind, wird die Länge des großen Rechtecks so reduziert, dass dessen Länge nicht höher als die Gesamtlänge der Ebenen ist.

Die Breite des umgebenden Rechtecks kann zwischen der maximalen Breite eines einzelnen Rechtecks und der Gesamtbreite aller Rechtecke variiert werden. Aus Gründen der Einfachheit führe ich das oben beschriebene Platzierungsverfahren für diese und alle Breiten dazwischen aus.

\section{Umsetzung}

\section{Beispiele}

\section{Quellcode (ausschnittsweise)}

\printbibliography 


\end{document}
